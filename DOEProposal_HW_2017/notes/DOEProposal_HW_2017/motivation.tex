\section{Physics motivation\label{sec:physHI}}
\label{sec:physics}

{\bf [Yen-Jie working on this section]}

In this Section we discuss the physics performance of the upgraded L1 trigger system for the 2018 high luminosity PbPb run using three examples of measurements using rare probes. 
The performance of the upgraded system, which uses background subtraction for L1 event rejection and provides 
access to the full expected luminosity of 3~nb$^{-1}$ is compared to that of the current system. 
These examples illustrate the large improvements in statistical precision afforded by the  improved 
ability to select jet events and high \pt\ particles during high luminosity PbPb collisions. 


%Studies of high \pt\ jet and charged particle production in the 2010 and 2011 PbPb data sets have helped to elucidate the nature of parton energy loss in hot nuclear matter. Parton energy loss was characterized using a wide range 
%of different signatures, ranging from the the dijet energy imbalance in central collisions and the suppression of single particle
%spectra at the highest \pt\ to modifications of jet shapes and fragmentation patterns~\cite{Chatrchyan:2011sx,Aamodt:2010jd,Alice:dihadron,CMS:2012aa,Chatrchyan:2012nia,Aad:2010bu,atlas:2012is,Chatrchyan:2012gt,Chatrchyan:2012gw}. 

Below we discuss a selection of measurements which exploit the future high luminosity data sets to 
provide important insights into the parton energy loss mechanism and heavy flavor particle production:

\begin{enumerate}
\item The flavor dependence of jet quenching is a crucial important test ground for various parton energy loss models. 
Compared to light quarks, heavy quarks are expected to suffer from smaller radiative energy loss when passing 
through the medium because gluon radiation is suppressed at angles smaller than the ratio of the quark mass $M$ to 
its energy $E$~\cite{Dokshitzer:2001zm}. This effect (and its disappearance at high \pt) can be studied using tagging
of b-decays and b-jets.
\item Compared to quarks, gluons are expected to suffer a larger 
energy loss because of the larger color factors. This effect can be studied using three-jet events.
\item Measurements of the single particle azimuthal anisotropy at very high \pt\ give access to the in-medium path-length 
dependence of energy loss. The azimuthal anisotropy as a function of charged particle \pt\ also yields 
important information about the parton energy dependence of the energy loss. At very low~\pt\ region thcxx
\end{enumerate}

These and other measurements rely on a sufficiently selective jet and heavy flavor meson trigger to provide access to the 
full delivered luminosity and a large minimum-bias sample.  Data collected during the 2015 PbPb run were used to estimate physics reach  
for a future high-luminosity run (assuming $L_{int} =$3~nb$^{-1}$) for these three physics cases. 
%Using typical thresholds of leading jet \pt\ $>100$~GeV/c, Figure~\ref{fig:efficiency_comparison} shows that 
%with the upgraded system (labeled ``Stage-1 system'' in the plots) sufficient rejection factors can be 
%achieved to sample the full delivered luminosity. For the current system (labeled ``Current system'' in the 
%plots), the rejection factor at L1 is limited to about two, requiring a large prescale factor to be applied
%due to bandwidth constraints in the detector front end readout. Figure~\ref{fig:aj_2015} to 
%\ref{fig:xT_scaling} show the effect of the resulting difference in statistical reach 
%for the physics examples discussed below.

\begin{figure}[!ht]
\begin{center}
%\includegraphics[width=.60\textwidth]{fig/heavyIon//efficiency_comparison_l1primitives.pdf}
\caption{The jet finder is applied to minimum bias data from 2011 without
event selection using the L1 primitives present in the RAW data. The
fraction of events which have a jet above a given $E_t$ threshold is plotted
as a function of that threshold. Note that the threshold is applied to the
L1 jet energy and does not correspond to the HLT or offline jet energy
scales.}
\label{fig:efficiency_comparison}
\end{center}
\end{figure}


\subsection{Nuclear Modification Factors of Heavy Flavor Mesons}
{\bf [To be updated]}

One of the proposed observables that reveal the flavor dependence of in-medium parton energy loss is the reduction of heavy flavor meson yield. This can be studied by measurements of nuclear modification factors (RAA), defined as the ratio of the yield in nucleus-nucleus collisions to that observed in pp collisions, scaled by the number of binary nucleon-nucleon collisions. Examples of theoretcal calculations of $R_{AA}$ for D and B mesons are shown in Figure~\ref{}. 

\begin{figure}[!ht]
\begin{center}
\includegraphics[width=.90\textwidth]{figures/cTheoryRAA_BD_v1.pdf}
\caption{Theoretical calculations of nuclear modification factors of charged particles, $D^0$ and $B^+$ mesons.}
\label{fig:aj_2015}
\end{center}
\end{figure}


\begin{figure}[!ht]
\begin{center}
\includegraphics[width=.90\textwidth]{figures/cRAA_lumiTG_3_lumiMB_1_v2.pdf}
\caption{Nuclear modification factors of charged particles, $D^0$, $D_s$ and $B^+$ mesons with 2015 data (left panel) and the statistics expected with L1 trigger rate upgrade.}
\label{fig:RAA_2015}
\end{center}
\end{figure}

\subsection{Ellipic Flow of Heavy Flavor Mesons}


\begin{figure}[!ht]
\begin{center}
\includegraphics[width=.90\textwidth]{figures/cTheoryV2_D_v1.pdf}
\caption{Theoretical calculations of v2.}
\label{fig:aj_2015}
\end{center}
\end{figure}


\subsection{$D^0$-Jet and $D^0$-hadron correlations}

\begin{figure}[!ht]
\begin{center}
%\includegraphics[width=.60\textwidth]{fig/heavyIon//r32_2015.pdf}
\caption{The ratio of 3-jet to 2-jet events ($R_{32}$ )as a function of the
average \pt\ of the two leading jets for \pt\ $> $ 100 GeV/c 
in the ten percent most central collisions expected in the 2015 PbPb Run.}
\label{fig:r32_2015}
\end{center}
\end{figure}

\begin{figure}[!ht]
\begin{center}
%\includegraphics[width=.50\textwidth]{fig/heavyIon//HighPtTrackReach_xTscaling_20121216.pdf}
\caption{Track \pt\ distribution for 2.76 TeV PbPb in 2011 with a total
integrated luminosity
         of 150$\mu$b$^{-1}$ (solid black), projection to 5.5 TeV in 2015
based on $x_T$ scaling 
         without upgraded L1 trigger (open red), and with upgraded L1
trigger (solid red).}
\label{fig:xT_scaling}
\end{center}
\end{figure}

\begin{figure}[!ht]
\begin{center}
%\includegraphics[width=.48\textwidth]{fig/heavyIon//JetvsTrack_PU1_centmin0_centmax12_20130214.pdf}
%\includegraphics[width=.48\textwidth]{fig/heavyIon/efficiency_trackpt_PU1_centmin0_centmax12_20130214.pdf}
\caption{Leading L1 jet $E_T$, with upgraded L1 system, vs leading track
\pt\ in 0-30\% PbPb collisions
at 2.76 TeV (left), and efficiency turn-on curve as a function of leading
track \pt\ for L1 jet $E_T$
thresholds of 40, 50 and 60 GeV (right).}
\label{fig:trigEff_track2015_central}
\end{center}
\end{figure}

\begin{figure}[!ht]
\begin{center}
%\includegraphics[width=.48\textwidth]{fig/heavyIon//JetvsTrack_PU1_centmin16_centmax40_20130214.pdf}
%\includegraphics[width=.48\textwidth]{fig/heavyIon//efficiency_trackpt_PU1_centmin16_centmax40_20130214.pdf}
\caption{Leading L1 jet $E_T$, with upgraded L1 system, vs leading track
\pt\ in 40-100\% PbPb collisions
at 2.76 TeV (left), and efficiency turn-on curve as a function of leading
track \pt\ for L1 jet $E_T$
thresholds of 40, 50 and 60 GeV (right).}
\label{fig:trigEff_track2015_peripheral}
\end{center}
\end{figure}

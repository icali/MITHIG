\section{L1 trigger\label{sec:L1Trigger}}
For the 2015 Pb-Pb run an upgraded L1 trigger system was designed. The Stage-1 L1 trigger was operated with specific HI
algorithms. However, in the middle of 2016 the full trigger upgrade (referred as Stage-2 in previous proposals) was
installed and commissioned. The upgraded trigger system has a new layer-1 and the single MP7 board layer-2 was replaced
with 9 MP7 boards operating with a time-multiplexed architecture. Also the firmware was completely redesigned by the L1
team including a series of new algorithm specific to p-p collision operation. In this optics, it is mandatory for the
2018 run to port to the new system the specific HI algorithm. In the list we have a specific background subtraction,
centrality triggers, single-track triggers and Q2 triggers. It was agreed with the L1 team that the stage-2 engineers
would take care of the actual firmware implementation and commissioning. However, it will be responsibility of the HI
group to provide support. It includes the performance studies of the algorithms and a specific description of the
implementation. Furthermore some of the several tests on the bench or at P5 will be performed by the HI group. Also the
compilation of the L1 and HLT menus will remain responsibility of the HI group. In term of manpower, it should be
considered a person for 6 months to follow the implementation process of the layer-2 and uGT firmware as well as
performing the performance studies needed for the algorithm implementation. The preparation of the L1 menu could be
considered as part of this time. 
=======
The electron/photon sort operation must determine the four highest transverse energy objects from 72 candidates supplied by the RCT, 
for both isolated and non-isolated electrons/photons.

To sort the jets the GCT must first perform jet finding and calibrate the clustered jet energies. The jets are created from the 396 
regional transverse energy sums supplied by the RCT. These are the sum of contributions from both the hadronic and electromagnetic 
calorimeters.

\section{Introduction}
\label{sec:intro}
High-energy collisions of heavy ions allow experimental studies of quark-gluon plasma (QGP), the 
equilibrated high temperature state of de-confined quarks and gluons. 
At the Large Hadron Collider (LHC) the abundant production of hard probes such as vector bosons, jets and quarkonia 
has opened a new era in the characterization of QGP, providing new information on initial state properties, parton energy loss
and color screening.  The capabilities of the Compact Muon Solenoid (CMS) detector in the detection of 
high momentum photons, electrons, muons, charged particles and jets have proven to be a unique match for the 
opportunities at the LHC. A key aspect of this success has been the convergence of detector needs for precision 
measurements in pp discovery physics in a high luminosity, high pileup environment and for QGP studies 
in the high multiplicity PbPb collisions. 
This convergence has allowed CMS to adapt a large number of new analysis techniques to studies of heavy ion collisions,
such as particle-flow jet reconstruction, studies of jet-shapes, lifetime fits for secondary $J/\psi$ studies, 
b-tagging of jets and missing $p_T$ measurements of W production and energy flow in dijet events. 

The full power of these and other techniques will be exploited in future heavy ion studies beginning in 2015,
when an increase in the collision energy to $\sqrt{s_{_{NN}}} = 5.5$~TeV and an eventual increase in PbPb collision rate 
to as high as $\sim 30$~kHz \cite{Jowett:2012priv} will enhance the production rate of hard probes by more than an order of magnitude. 
In this next era, CMS will undertake precision studies of complex observables such as $\gamma$-, Z$^0$- and W$^\pm$-jet correlations,
b-jet quenching, multi-jet correlations and the azimuthal anisotropy of high \pt\ jets and quarkonia. These and other measurements
will provide deep insight into the properties of strongly interacting matter at the temperature and energy density frontier.

The CMS trigger and data acquisition system is the key to achieving the ultimate high \pt physics reach in heavy ion collisions.
Uniquely, the CMS trigger system only has two main components, the hardware based Level-1 (L1) trigger and the High-Level Trigger,
which is implemented as "offline" algorithms running on a large computer farm with access to the full event information. 
The conceptual aspects of applying this system 
to heavy ion collisions, which are characterized by much lower rate, but much higher multiplicity than those encountered in 
pp collisions, were first developed by the US CMS HI group (see, e.g.\ \cite{Roland:2007is}). The HLT trigger work 
formed the basis for our 
previous proposal to the DOE Office of Nuclear Science. As we demonstrated in that proposal, triggering on high \pt probes for
PbPb collision rates up to several kHz (i.e.\ the design value for PbPb) is possible using event rejection 
solely or mostly at the HLT level. In the 2011 PbPb run, this allowed a reduction of the event rate to storage by more than
an order of magnitude compared to the collision rate, without prescaling the most interesting high \pt\ observables.


It has now become clear that the LHC will be able to significantly exceed the PbPb design luminosity in future runs,
possibly reaching up to 30~kHz already in 2015. This will place an even greater 
emphasis on the CMS trigger system. Bandwidth constraints in the front end detector readout in PbPb ccollisions will
require a change in the trigger strategy, requiring significant rejection factors
already in the L1 trigger. For the unbiased single track and jet triggers, that are critical for the CMS physics program, 
necessitates the subtraction of the underlying event background for the L1 trigger decision to achieve
L1 rejection factors greater than the current factor of two (in the current system, the background subtraction
for jet triggers is done at the HLT level).

Experience from the previous runs shows that a factor of 10-20 rejection is achievable if the background subtraction 
is performed using the full $\phi$ information to estimate the background level at a given $\eta$ position 
of a jet candidate. The hardware layout of the current L1 system precludes the implementation of such background 
subtraction algorithms. 

In this proposal, we will present the CMS L1 upgrade design and demonstrate that the planned upgrade delivers 
the needed capability to record high \pt\ jets and charged particles at the full delivered rate in high luminosity 
PbPb runs in 2015 and beyond.  This new system will need to be commissioned by 2015, well before the next heavy ion running period,
to ensure successful PbPb data taking.  

Section~\ref{sec:physics} describes the expected performance of the upgraded 
trigger for several examples of critical measurements, compared to the the limitations of the current system. 
The CMS trigger system and the technical details of the proposed upgrade project are described in Section~\ref{sec:calo}. 
In Section~\ref{sec:hiTrigPerf} the trigger performance based on simulation and extrapolation are illustrated. Finally, 
the proposed schedule, management plan and the budget are described in Sections~\ref{sec:management} and~\ref{sec:funding}.


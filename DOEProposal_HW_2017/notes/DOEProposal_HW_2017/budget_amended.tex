\section{Budget}
\label{sec:funding}

The overall budget for the heavy-ion group contribution to the L1 trigger upgrade includes cost of production of the optical communication boards oRSC and related optical fiber and patch panels as well at the support for one postdoctoral associate and the cost of living adjustment for two graduate students stationed at CERN. The collaborating groups are funded directly from other sources.

The oRSC board is a VME board that contains a Spartan-6 which is a VME slave and I2C master for local devices, and a Kintex-7, which is a 80 MHz parallel to optical SERDES bit packer. The boards will be designed, produced and assembled by the Wisconsin group and then tested, commissioned and installed by the MIT, Rice and Wisconsin team at CERN. The estimate cost for one board is 6K\$ where 2.5K\$ is for FPGA, 1K\$ for optical elements and 2.5K\$ for assembly and production. To fully equip trigger system we need 18 boards. In addition we need spare boards to assure uninterrupted operation of the experiment. The standard practice in CMS is to order additional 4 spare boards (20\% of the total). There is also a plan to first produce 3 prototype boards. The total cost of the 3 prototype boards is 18K\$ and the cost of final 22 boards is 132K\$. Since the boards are still in the prototyping stage we assumed an additional 20\% contingency on the electronics cost.

The testing and commissioning of the oRSC boards and the MP7 firmware will take place at CERN. The development of the firmware on the MP7 modules will start already in spring of 2013. Tests of prototypes of the oRSC boards will take place during summer of 2013 and the production and testing of the final boards will take place late in fall of 2013 and early 2013. The installation of the complete system is scheduled for summer of 2014. CMS trigger groups have large electronics testing and repair facility at CERN with the ability to operate full DAQ chain independently of the detector operation. It would be very difficult if not impossible to replicate such system at home institutions of heavy-ion participants. In addition, the experience developed during testing will be applicable to the installation and commissioning in the experiment. This is why the group of people involved must be stationed at CERN in Geneva.

The members of the team from the heavy-ion groups will consist of an MIT postdoctoral associate and a graduate student. The fully loaded off-campus base salary of the MIT postdoc is presently ~87K\$. To compensate for the CHF/\$ exchange rate and the higher cost of living in Geneva we will provide Cost of Living Allowance to the postdoc and Per Diem to the graduate students. The exact amount depends on the exchange rate at the time of payment but we estimate it to be about 35K\$ for the postdoc and 15K\$ per student per year.  The activities of the postdoc and the graduate students will ramp up during the spring and be full time starting in summer of 2013. In addition there is some amount allocated for travel to and from Geneva. The costs is summarized in Table~\ref{OpCost} and they all include MIT overhead.

\begin{table}[hbt]
\begin{center}
\begin{tabular}{|c|r|r|r|r|r|r|r|}
\hline
Year        & Salary & COLA & Equipment & Travel & Travel& Total \\ \hline
            &  1 postdoc   & 1 postdoc   &  & Per diem       &  Add. Travel     &     \\ \hline
7/1/13-6/30/14  &     &    &       &   &      &    \\ \hline
\end{tabular}
\end{center}
\caption{L1 trigger upgrade funding profile, in K\$}
\label{OpCost}
\end{table}

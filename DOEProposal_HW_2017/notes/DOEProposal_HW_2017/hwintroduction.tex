\newpage

\section{Hardware Introduction\label{sec:HWintro}}
The CMS experiment is mainly designed to record high luminosity, high-PU proton-proton collisions events.  However, with
several detectors tweaks/configuration changes, over the last several years, the CMS experiment has been able to record
successfully also the high multiplicity Heavy-Ions collisions delivered by LHC. As discussed already in previous
proposals, the main changes required to allow CMS to operate in HI environments could be summarized as follows:

\begin{itemize}
\item Dedicated Silicon Pixel Front End device firmware for high multiplicity environment. 
\item The Silicon Tracker zero suppression were bypassed and all the channels information were forwarded to the CMS DAQ system. The Zero Suppression was performed in HLT. A new tracker FED firmware was also designed to reduce the event payload of ~30% without information loss. 
\item DAQ reconfiguration / rebalance to deal with big data volumes and high throughput. In standard p-p collisions operation, an average event size correspond to ~ 1- 1.5 MB/even. During Pb-Pb collision operation, the average event size is at around 17 MB/event. 
\item The Level-1 trigger firmware was redesigned to perform a dedicated underlying event subtraction algorithm and to include specific HI algorithms 
\item Significant adjustment in the overall CMS dataflow
\end{itemize}


With the specific configuration described above and with a series of small adjustments not mentioned here, CMS was able
to run smoothly during the 2015 Pb-Pb data-taking with and average L1 rate of 10 kHz. Few hours of the Pb-Pb data-taking
period were also devoted to explore the limit of the system.  With the 2015  CMS configuration for Pb-Pb, the absolute
L1 rate limit is of 12 kHz. During the 2015 run, the collision rate was of ~20 kHz. For the 2018 Pb-Pb run and beyond
the rate limits could significantly impact the expected physics program. On one hand it would be beneficial to increase
the number of min bias events collected and on the other hand LHC will deliver a higher luminosity. We are considering a
collision rate of ~30 kHz already for 2018 Pb-Pb run. The CMS subsystems as well as their interaction with the DAQ have
been studied and a series of bottlenecks were identified. In this proposal the main interest area are specified
reporting also about the corresponding manpower needs. The preliminary goal is to increase the CMS readout rate to ~30
kHz. It is however important to bear in mind that this proposal will focus only on aspects that will potentially require
hardware changed.  The comprehensive series of tasks needed for the overall run preparation is not reported here. 


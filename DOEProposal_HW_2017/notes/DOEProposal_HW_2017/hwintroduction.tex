\subsection{Introduction\label{subsec:HWintro}}
The CMS experiment is mainly designed to record high luminosity, high-PU proton-proton collisions events.  However, with several detectors tweak/configuration changes, over the last several years, the CMS experiment has been able to record successfully also the high multiplicity Heavy-Ions collisions delivered by the LHC. As discussed already in previous
proposals, the main changes required to allow CMS to operate in HI environments could be summarized as follows:

\begin{itemize}
\item A dedicated silicon pixel Front End Device (FED) firmware for high multiplicity environment. The overall FED data flow has been re-designed changing the readout scheme of the input links and a redistribution of internal buffers.
 
\item The silicon tracker zero suppression was bypassed and all the channels information were forwarded to the CMS DAQ system. The Zero Suppression was performed in HLT. A new tracker FED firmware was also designed to reduce the event payload of $\sim 30 \%$ without information loss. The reduction was obtained stripping out from the standard data format unused information.  

\item DAQ reconfiguration / rebalance to deal with big data volumes and high throughput. In standard p-p collisions operation, an average event size corresponds to $\sim 1-1.5$ MB. During Pb-Pb collision operation, the average event size is at around 17 MB. 

\item The Level-1 trigger firmware was redesigned to perform a dedicated underlying event subtraction algorithm and to include specific HI algorithms.

\item Significant adjustment in the overall CMS data-flow.
\end{itemize}


With the specific configuration described above, and with a series of other small adjustments not mentioned here, CMS was able to run smoothly during the 2015 Pb-Pb data-taking with an average L1 rate of $\sim 10$ kHz. Few hours of the Pb-Pb data-taking
period were also devoted to exploring the limit of the CMS in term of maximum L1 rate allowed.  With the 2015 CMS configuration for Pb-Pb, the absolute L1 rate limit was observed of 12 kHz. In the next sections, this limit, as well as solutions to increase it, will be discussed. 

During the 2015 run, the collision rate was of $\sim 20$ kHz. For the 2018 Pb-Pb run it is expected an increase of collision rate up to $\sim 30$ kHz. Without applying any changes to the L1 trigger mix used in 2015, this extra factor would mean to operate CMS at above its limits with a significantly impact on the physics program. In addition, as discussed in the previous sections, it would be beneficial to also further increase the number of minimum bias events collected. 

The CMS subsystems, as well as their interaction with the DAQ, have been studied and a series of bottleneck areas were identified. This proposal addresses these limitations. The goal is to increase the CMS readout rate to 30 kHz. It is, however, important to bear in mind that this proposal will focus only on aspects that will require
hardware changes. The comprehensive series of tasks needed for the overall run preparation is not reported here. 


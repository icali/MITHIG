\newpage

\section{Trigger Upgrade Goals and Plans\label{sec:calo}}

In this section we provide a recall the details and limitations of the present CMS PbPb readout scheme and L1 trigger system 
and discuss the technical implementation of the Stage-1 and Stage-2 phases of the L1 upgrade project. 

One key difference between CMS readout in PbPb compared to pp running relates to the strip tracker.
In PbPb running the strip tracker of CMS is read out in ``Virgin Raw mode'' which corresponds to 
reading out all the 10~M silicon strips without any zero suppression. This allows execution of very precise zero 
suppression algorithms, that are very difficult to implement in the Front End Controllers, either offline (as was done for 2010 PbPb
data taking) or at the HLT (as was done in 2011). This is critical for assuring good track reconstruction efficiency 
in the high track density environment created in central PbPb collisions.  
Due to the tracker/DAQ readout bandwidth limitations, the maximum achievable L1 readout rate during PbPb collisions 
in Virgin Raw mode is about 3 kHz. Independently, the pixel readout chain sets a limit to the maximum L1 accept 
rate of similar magnitude due to the risk of buffer overflow in the case of 2 subsequent PbPb events with very 
high particle multiplicity. 

The limit of the readout rate, in combination with the limited selectivity of the current L1 system for jet triggers 
of about two, sets the maximum collision rate that the present CMS experimental setup can fully exploit to about 6kHz.
In the 2011 PbPb run the accelerator achieved collision rates of about 4.5 kHz, well within the limits of the present system.

To exploit the expected increase in the luminosity after the current shutdown (LS1), we propose to participate in 
the planned upgrade of the L1 trigger system, allowing the upgrade plan to be accelerated to match the PbPb needs. 
CMS is planning to upgrade the L1 trigger system in two stages. Stage-1 will occur before the 2015 HI run.  Stage-1
will re-use most of the existing triggering infrastructure with important improvements in the communication 
between different detector regions as well as upgrades to the readout of trigger primitives from the forward 
calorimeter, HF. This is achieved by creating a parallel system with new electronics acting as a data concentrator 
and receiving information from all the trigger primitives of the system. 
The Stage-1 upgrade will be sufficient to handle the rates expected before the next long shutdown in 2018 (LS2). 
Stage-2 will be combined with the more extensive upgrade of the HCAL and ECAL readout and will further increase 
the speed and granularity of the information available to make L1 trigger decisions. 


The electronics for the Stage-1 and Stage-2 systems will be based on the latest FPGAs
that use high speed serial links (5-10
Gb/s) to provide high bandwidth I/O.  While providing Tb/s of I/O per
FPGA, serial links do require $\sim$100~ns to serialize and
de-serialize the data and thus the number of layers in the trigger
should be kept to a minimum to minimize latency. Consequently, the
system comprises just 2 layers, with 2 processing card types based on
Xilinx Virtex 7 FPGAs. 

In order to minimize the size, cost and latency of the system the HCAL 
and ECAL Trigger Primitive Generators (TPGs) must also support 5-10 Gb/s serial links.
Furthermore, at these link speeds and density the signals must be transmitted optically. 
The trigger primitive input to the present system is based on 1.2 Gb/s serial links over copper.  
For ECAL almost all the current electronics will remain unchanged, 
but the Serial Link Board (SLB) that performs
the parallel to serial conversion will be replaced with an optical
equivalent (oSLB - optical SLB), with 4.8 Gb/s output. The ECAL readout upgrade will occur during LS2 in time for Stage-2 upgrade.
For Stage-2 the HCAL project will upgrade the so-called HCAL ``back-end
electronics'' (trigger and readout cards) during LS1 and after the
first year of running post-LS1, and thus will be able to easily
provide optical signals because the upgrade is based on the same
technology.\\

An additional feature of the oSLB is a second output which will, in
conjunction with the replacement of the equivalent card on the
existing Regional Calorimeter Trigger (RCT, described in the following
section), allow the present system to operate in parallel to the
new system. HCAL will also be able to drive both the new \& old
systems, but crucially the duplication to run both systems appears
earlier in the HCAL trigger path by splitting the optical fiber input
to the back-end electronics.
Thus the new and existing HCAL back-end electronics will both be present in the counting room.  The
exception to this is HF which will only have the new back-end
electronics and thus RCT input will need to be upgraded to handle the
new optical inputs for parallel operation. The ability to run both old 
and new system in parallel ensures that the upgrade does not jeopardize 
the trigger at a crucial time for CMS.

The Stage-1 of upgrade together with elements of Stage-2 will be commissioned 
in 2015 in parallel with running the current trigger. 
Both old and new trigger will be operated during the pp run such that they can be fully debugged. 
The heavy-ion run is expected after almost a year of pp running, giving sufficient time
to prepare.

In Stage-1 the the existing RCT will be kept, but the GCT will be replaced with prototype hardware.
The system will operate with a coarser and fixed granularity than the full Stage-2 upgrade,
but will be sufficient to improve the jet energy resolution and the underlying event
subtraction as shown below.

The calorimeter trigger upgrade will improve both the quality and reduce the rate of key physics objects.  It will provide:

\begin{enumerate}
\item Improved pileup subtraction; information from regions for the full $\phi$ and $\eta$ range will be available at the time of 
decision. A pileup subtraction algorithm based on information from  full $\phi$ rings in slices of $\eta$ will be available. 

\item Improved energy resolution; from both the better construction of objects (e.g. objects will have pile up subtraction applied), but also from improved trigger primitives as the HCAL upgrade is implemented.

\item Additional flexibility; the current system is comprised of many cards that each typically perform a single task with limited sharing of data between cards.  Inevitably this limits the algorithms that can be implemented. This will be improved in the upgraded design.
\end{enumerate}

The following Section describes further details of the present system and of the changes that will be implemented as a part of the 
Stage-1 upgrade.

\subsection{Present System\label{calo:present}}

\subsubsection{Overview}

The present CMS calorimeter trigger is built in two stages, a Regional 
Calorimeter Trigger (RCT) and Global Calorimeter Trigger (GCT).  The 
RCT receives Trigger Primitives (TPs) in the form of an 8-bit rank (E$_T$) 
and quality bit from over 8000 towers in the Hadronic (HCAL), Electromagnetic 
(ECAL), and Forward (HF) Calorimeters for trigger coverage out to $|\eta|<5$.  
The RCT processes this information in parallel and sends 18$\times$8 
e/$\gamma$ candidates, 36 E$_T$ sums of 4$\times$4 tower regions, 144 
HF region E$_T$s, and corresponding bits for object classification.  The 
GCT sorts the e/$\gamma$ candidates further, finds Jets (Central, Forward, 
and Tau) using the sums and HF E$_T$s, and calculates global quantities like 
Missing E$_T$. It then sends four e/$\gamma$ each of two types, isolated and 
non-isolated, four each of Central, Tau, and Forward Jets, Jet counts, and 
several global quantities to the CMS Global Trigger (GT) for the final CMS 
L1 trigger decision.

Data from the ECAL front-end is received on fibers at the front panels 
of the ECAL Trigger Concentrator Cards (TTC) in the CMS underground counting 
room (USC55) and converted to Trigger Primitives for 4032 electromagnetic 
calorimeter trigger towers over the entire ECAL Barrel (EB) and ECAL Endcap 
(EE).  In the barrel, a trigger tower is defined as 5$\times$5 crystals in 
the ECAL of total dimension 0.087$\times$0.087 ($\Delta\phi\times\Delta\eta$), 
corresponding 1:1 to the physical tower size of the HCAL.  In the endcap 
the towers are more irregular, but are matched as close as possible to the 
size of the corresponding HCAL tower.

Similarly, the data from the HCAL and HF front-end is received on fibers at 
the HCAL Trigger Readout (HTR) boards in USC55.  After processing, up to 6 
SLBs send the 4032 HCAL Barrel and Endcap trigger primitives and the 144 HF 
trigger primitives from HTRs to the RCT.  For each HCAL trigger tower an 
8-bit E$_T$ and a MIP bit are sent to the RCT.  For HF an 8-bit E$_T$ and 
a threshold-based bit are sent to the RCT.  In the HF the tower size is 
significantly larger, and currently is 0.5 in $\eta$ by 20$^\circ$ in $\phi$, 
a combination of up to 6 physical HF towers.

\subsubsection{RCT}

The  Regional Calorimeter Trigger (RCT) receives the TPs from the ECAL, HCAL, 
and HF at eighteen 9U crates of custom electronics.  Eight Receiver Mezzanine 
Cards (RMCs) with the V2716-1 mounted on it are on the RCT Receiver Cards 
(RCs) to receive HCAL and ECAL data.  A single mezzanine card located on 
the Jet/Summary Card is for the HF data.  Before processing, the data is 
deskewed, linearized, and aligned. 

Each RC processes 64 towers of data, 32 ECAL and 32 HCAL.  This is split 
into two 4$\times$4-tower regions for processing.  For the jets and global 
quantities, two 4$\times$4 tower sums (ECAL and HCAL summed per tower), an 
OR of the 16 corresponding MIP bits, and $\tau$-veto bits are made. For 
the electron/photon finding,  data is copied and shared, and aligned before 
transmission on a 160 MHz ECL custom backplane to 7 Electron Isolation Cards 
and one Jet/Summary Card.  The Electron Isolation Cards find 4 electron 
candidates,  one each isolated and non-isolated per 4$\times$4 trigger 
tower region out to $|\eta|<3$, using the shared data between the crates' 
Receiver Cards on cables, and from adjoining cards to ensure seamless 
coverage.  The Jet/Summary Card receives the HF data, forwards it to the 
GCT, does the final electron/photon sort, and sends the regional E$_T$ 
sums and the electron candidates to the Global Calorimeter Trigger (GCT).  
For the purpose of the heavy-ion physics primarily the jet and electron/photon candidates are selected.


\subsubsection{GCT}

The Global Calorimeter Trigger (GCT) is the last stage of the L1 CMS calorimeter trigger chain. A detailed description of the GCT design, implementation and commissioning is beyond the scope of this paper and is provided in several conference papers 
~\cite{gct1,gct2} describing the changes in design since the CMS Trigger TDR~\cite{cms_triggerTDR}. The trigger objects computed by the GCT from data supplied by the Regional Calorimeter Trigger are listed below and described 
in subsequent paragraphs:

\begin{itemize}
\item four isolated and four non-isolated electrons/photons of highest transverse energy; 
\item four central, four forward and four tau jets of highest transverse energy; 
\item total transverse energy; total jet transverse energy;missing transverse energy; 
\item missing jet transverse energy; summing of feature bits and transverse energies in the forward hadronic calorimeter.
\end{itemize}

The electron/photon sort operation must determine the four highest transverse energy objects from 72 candidates supplied by the RCT, 
for both isolated and non-isolated electrons/photons.

To sort the jets the GCT must first perform jet finding and calibrate the clustered jet energies. The jets are created from the 396 
regional transverse energy sums supplied by the RCT. These are the sum of contributions from both the hadronic and electromagnetic 
calorimeters.

In addition to these tasks the GCT acts as a readout device for both itself and the RCT by storing information until receipt of a L1A 
and then sending the information to the DAQ.



\subsection {Stage-1 trigger upgrade\label{calo:stage1}}

The current calorimeter trigger will be modified, essentially replacing 
the GCT with prototype hardware. The RCT will
be reprogrammed to provide two types of clusters, 4x4-clusters and
2x1-clusters.  Each RCT crate reports all fourteen 4x4-clusters, which
are processed by it as 10-bit $E_T$ sums of both ECAL and HCAL
response and an identification bit, to indicate if the energy is
electromagnetic or not.  Each RCT crate reports four highest $E_T$
2x1-clusters, which are reported as 6-bit $E_T$ sums of ECAL and HCAL
response.  These 2x1-clusters could be locally isolated using the
nearest neighbor ``L-isolation'' algorithm implemented in the current
RCT. The schematics of the Stage-1 system is shown in Figure~\ref{fig:L1_Stage1}.

The modern $\mu$TCA processors with latest FPGAs (MP7 cards) with
significant resources will be used in the layer-2 system to receive
all the RCT data in one card to implement the improved algorithms
described above.  The cluster data from the current RCT crates
(layer-1) is converted from its present 80-MHz differential ECL copper
links to an optical format (6.4 Gbps) using a new VME card, the oRSC.

Eighteen oRSC, one per RCT crate, are needed to serve all the data
to the new layer-2 hardware.  The data is sent on two 6.4 Gbps fibers
per oRSC. The oRSC is also able to replicate these two fibers worth
of data six times.  This feature enables the use of separate 
cards for e$\gamma\tau$, $\mu$ and jet paths enabling significant
FPGA resources for processing.  It is also planned that one of the cards will be dedicated to heavy ion
selection processing. 

The oRSC receives data from the neighboring Jet Summary Card in the
RCT crate on cables at 80 MHz in differential ECL format. The oRSC
also receives timing information from the RCT clock fanout system.
The data is appropriately conditioned and fed into a Xilinx Kintex
XC7K325T chip which has 24 serial drivers operating at up to 6.4 Gbps.
The firmware will be configured to multiplex the data and serve it out
in serial format on these outputs.  The logic cells available on this
chip are also available for pre-computation of some sums, e.g., for
pileup level determination if needed.

For the forward region of the calorimeter (HF), finer granularity 
information is available in optical format.  Therefore, a new layer-1 
system is implemented using two CTP7 cards.  The clusters with finer
position resolution are used in jet finding in this region. 

\begin{figure}
\begin{center}
\vspace{.2in}
\centerline {
%\includegraphics[width=4in]{fig/heavyIon/L1_schematics.pdf}
}
\vspace{.2in}
\end{center}
\caption{Schematics of the Stage-1 upgrade.
\label{fig:L1_Stage1} }
\end{figure}

%The layer-2 system computes the pileup contribution event-by-event
%using the 4x4 information.  Algorithms could use a truncated mean
%or statistical median or mode to determine the pileup contribution.
%The pileup contribution will be subtracted from the isolation and
%jet regions.

%The traditional trigger architecture is available as a fallback option
%for the full Stage-2.  In this case twenty four CTP7 cards will be
%used with nearest neighbor sharing to obtain all particle clusters.
%The advantage in Stage-2 would be that the layer-1 clusters can be
%reported at higher position and energy resolution than in the current
%RCT system.  These finer clusters will be transferred to layer-2 cards
%for finding the trigger objects as described above for Calo-2015 system.

%Mixed use of RCT for HCAL clusters and twelve CTP7 cards for
%electromagnetic clusters provides an option to implement a good
%alternative for the electron/photon/tau triggers, within the
%traditional architecture, although the jet and energy objects remain
%somewhat constrained due to the 4x4-cluster size from RCT.

%\subsection{oRSC}

\subsubsection{Upgrade to HF trigger primitives\label{calo:hf}}

Readout of HF calorimeter is expected to be upgraded in time for Stage-1, during LS1 shutdown.
The \utca electronics is expected to operate with both the
existing RCT trigger and the upgraded calorimeter trigger
simultaneously. Each $\mu$HTR will be connected to a single $\Delta\phi =
20^\circ$ wedge of either HF+ or HF-.  For the existing RCT trigger,
the $\mu$HTR is responsible for calculating region energy sums which cover
$\Delta \eta \times \Delta \phi = 0.52 \times 0.35$.  With
single-anode PMTs, this corresponds to a sum over twelve channels, six
long-fiber channels and six short-fiber channel.  

For the HF link to the upgraded trigger, the data will be transferred
over two fibers per $\mu$HTR at 6.4 Gbps using 8b10b encoding.  The link
will thus carry sixteen bytes from the $\mu$HTR to the calorimeter trigger
for every bunch crossing.  

Each trigger
primitive will consist of an eight-bit \ET value determined from the
combination of the long and short fiber energy for the tower and two
feature bits.  The \ET may be compressed to a non-linear scale by a
look-up table which can be undone at the trigger to allow for a broader
dynamic range while retaining the ability to distinguish small
signals.  One of the feature bits will indicate if the distribution of
energy in the tower between the long and short fibers is consistent
with an electron.  The second feature bit will indicate energy deposits which
are out-of-time with respect to the collision, as determined by the
TDC measurements.

The EM feature bit and the full-granularity tower data will be
available to an upgraded calorimeter trigger as soon as the HF
back-end upgrade is complete -- the installation of the upgraded
front-end is not required.  The upgrade of the front-end will bring in
the TDC feature bit and new handles for identifying and recovering or
suppressing anomalous noise signals at the trigger level.  In
particular, the TDC readings and the energy distributions between the
various readouts will allow the identification of anomalous hits.  If
the anomalous hit affects only one of two anodes for a channel, the
response from the 'clean' channel will be doubled and sent for the
total energy response of the tower, with only a minimal loss of
resolution.  If the anomalous signal affects both anodes, as
determined from the TDCs, the tower energy will be zeroed in the $\mu$HTR.
If a trigger on out-of-time energy is defined, the $\mu$HTR can instead
pass the energy and use the TDC-out-of-time feature bit to identify
the situation.

As each $\mu$HTR has twelve fiber links available for transmitting trigger
data, there is also significant scope for expansion if required in the
future.

\subsubsection{Optical Synchronization and Link Boards (oSLBs) \& Optical Receiver Mezzanines (oRMs)}

The current calorimeter trigger uses a synchronization method
implemented in the synchronization and link board (SLB). These boards
allow the synchronization of electromagnetic and hadronic trigger
primitives at the LHC frequency (40.08 MHz) and its transmission to
the Regional Calorimeter Trigger. The data are received by Receiver
Mezzanine (RM) cards mounted on the RCT.   

At present each SLB transmits up to 4 channels at 1.2 Gb/s on copper.  The upgraded SLB (Optical SLB - oSLB) will concentrate the data onto a single 4.8 Gb/s fiber, with a 2nd fiber output for transmission to the upgraded calorimeter trigger.  The Receiver Mezzanine (RM) card on the RCT will also need to be upgraded to an optical version (oRM) to run both the old \& new trigger in parallel.  The design of the oRM and oSLB are very similar.

\subsubsection{Backplane optimized processor (CTP7) \& Optical I/O processor (MP7)}
The CTP7 is an AMC card with a single Virtex-7 XC7VX690T FPGA with GTX links, dual SDRAM for dedicated DAQ and TCP/IP buffering, a Module Management Controller (MMC), and custom-designed power modules. On the front panel, the CTP7 uses 5 Avago AFBR-820B Receiver Modules, and two AFBR-810B Transmitter modules to provide 60 optical inputs and 24 outputs running at up to 10 Gbps. For back-plane communication, as used to share data across calorimeter regions for seamless regional reconstruction, the CTP7 has 12 bidirectional links to the backplane which can be run at up to 4.8 Gbps. Slots 2-5 and 8-11 have bi-directional links in a 2x4 array for handling boundaries within a processing region. Slots 6 and 7 have links to the upper or lower ranks of the processing array and are used for collating data for inter-crate communication. For completeness, the remaining links are routed to slots 1 and 12, although currently unnecessary.\\

The MP7 was designed to be a generic stream-processing engine. This design philosophy manifests itself most clearly in the distinctive data interface, which is all-optical and has symmetric transmit and receive bandwidths. Using a single interface for the sending and receiving of all data offers several distinct advantages: Primarily the board ceases to have a specific role and becomes a truly generic stream-processing engine. Application of the board to a specific task is no longer restricted by the bandwidth and compatibility of each type of interface, but only by the total bandwidth, whilst all specialization required for a task is contained within the programming of the board and the interconnections between boards. The MP7‚ optical interface is provided by up to six Avago MiniPOD transmitters and six Avago MiniPOD receivers. Each MiniPOD device provides 12 optical links running at up to 10.3 Gbps, giving the MP7 a total optical bandwidth of up to 740 Gbps in each direction. The MP7's processing capability is provided by a Xilinx Virtex-7 FPGA. The MP7 has 144 differential-pairs running at 10Gbps and many more running at the order of 1-2Gbps.  The main source of risk in the design of the MP7 was the sheer number and high speed of the optical links and, as such, the main test of the MP7 is the integrity of the optical links. The links have been validated by loop-back test, both as two sets of 24 links running PRBS-31, and all 48 links simultaneously running 8B10B. Running the links at 10Gbps the MP7 transfers 0.48Tb per second in each direction. To date, the MP7 has transferred in excess of an Exhibit of data without an observed error, giving a limit on the per-board bit-error rate of approximately $3\cdot 10^{-17}$.


\subsubsection{Algorithms\label{calo:algos}}

Higher granularity input data and more flexibility through modern high-speed electronics 
allow for improvements in the calorimeter trigger algorithms already at Stage-1.

In heavy ion events the underlying event (UE) depends very strongly on the centrality of collision and it is largely uniform in azimuth but varying strongly in pseudo-rapidity. The level of the UE contribution 
can be estimated in rings in regions at fixed $\eta$ while using full $\phi$ information. By using iterative baseline estimate
it is possible to estimate the UE level and find localized energy distributions corresponding to jets. 
This may be done based on 4x4 tower regions, 
which would be possible to implement in Stage-1 and improved later 
with higher granularity and improved jet definitions in the Stage-2 trigger.

\subsection{Stage-1 implementation\label{calo:integration}}

Stage-1 of the upgrade, for operation before all the optical TPGs are
available, is based on a traditional trigger architecture.  It uses
the reprogrammed RCT and the oRSC cards to provide 4x4 and 2x1
clusters of energy and identification bits in the $|\eta<3|$ region.
These are processed in the MP7 cards.  Eighteen oRSC cards in the
eighteen RCT crates each provide two fibers of input to each MP7 card.
Up to five MP7 cards can be used to implement the algorithms discussed
above.  We envision that one MP7 each is dedicated to the
e$\gamma\tau$, $\mu$-isolation, jet/sums and heavy-ion algorithm
implementation. One CTP7 card is also used to receive the sixth set of
outputs from the oRSC to provide a readout and testing facility for
the RCT.  Stage-1 also includes two CTP7 cards to receive 36 fibers
each from the HF $\mu$HTRs.  The clusters identified in the CTP7 cards
are passed to the MP7 card performing the jet/sums trigger for
further processing to identify the jets and include in the MET, MHT
and other sums.  The objects are sorted in the MP7 to select the
top four candidates of various type and are forwarded to the GT
for final trigger determination using oGTI link boards.

The traditional trigger architecture is chosen for the Stage-1
calorimeter trigger upgrade plan.  The current RCT is
used as layer-1 processor to form the energy clusters.  The RCT will
be reprogrammed to provide two types of clusters, 4x4-clusters and
2x1-clusters.  Each RCT crate reports all fourteen 4x4-clusters, which
are processed by it as 10-bit $E_T$ sums of both ECAL and HCAL
response and an identification bit, to indicate if the energy is
electromagnetic or not.  Each RCT crate reports four highest $E_T$
2x1-clusters, which are reported as 6-bit $E_T$ sums of ECAL and HCAL
response.  These 2x1-clusters could be locally isolated using the
nearest neighbor ``L-isolation'' algorithm implemented in the current
RCT.

The modern $\mu$TCA processors with latest FPGAs (MP7 cards) with
significant resources will be used in the layer-2 system to receive
all the RCT data in one card to implement the improved algorithms
described above.  
For the forward region of the calorimeter (HF), finer granularity 
information is available in optical format.  Therefore, a new layer-1 
system is implemented using two CTP7 cards.  The clusters with finer
position resolution are used in jet finding in this region.

The layer-2 system computes the pileup contribution event-by-event
using the 4x4 information.  Algorithms could use a truncated mean
or statistical median or mode to determine the pileup contribution.
The pileup contribution will be subtracted from the isolation and
jet regions.

Mixed use of RCT for HCAL clusters and twelve CTP7 cards for
electromagnetic clusters provides an option to implement a good
alternative for the electron/photon/tau triggers, within the
traditional architecture, although the jet and energy objects remain
somewhat constrained due to the 4x4-cluster size from RCT.

The Stage-1 system requires RCT information to be transferred to the
layer-2 processors (MP7s).  The RCT output is reformatted as optical using the optical Regional Summary Cards (oRSC).
Eighteen oRSC, one per RCT crate, are needed to serve all the data
to the new layer-2 hardware.  The data is sent on two 6.4 Gbps fibers
per oRSC. The oRSC is also able to replicate these two fibers worth
of data six times.  This feature enables the use of separate 
cards for e$\gamma\tau$, $\mu$ and jet paths enabling significant
FPGA resources for processing.  The oRSC also has another 10-channel
optical output operating at 1.6 Gbps to serve the data to the leafcards
of the GCT, thus replacing the source card system.

The oRSC receives data from the neighboring Jet Summary Card in the
RCT crate and the data is appropriately conditioned and fed into a Xilinx Kintex
XC7K325T chip. The firmware will be configured to multiplex the data and serve it out
in serial format on these outputs.  The logic cells available on this
chip are also available for pre-computation of some sums, e.g., for
pileup level determination if needed.

\subsection{Stage-1 Integration Steps}

 Since some of the Stage-1 upgrade items,
especially oSLB and oRM deployment, is disruptive extreme caution is
necessary before touching the working trigger hardware at P5.
Therefore, we put together several demonstrators at Wisconsin and
Prevessin to validate the proposed hardware and firmware solutions
before dismantling the system for upgrades at the P5.  Since the TPG
input and full 18-crate RCT system are not possible to replicate
elsewhere, we use some of the new cards in emulation mode to provide
input to later layers of hardware/firmware.  Stand-alone testing
begins at Wisconsin and validated hardware moves to Prevessin, where
partial integration with the TPG and MP7 systems is possible.  There
will be two ``full-system'' demonstrator systems, using emulated data
sources, available at both locations to develop and validate firmware.

Three prototype CTP6 cards will be configured to emulate the entire RCT.
These three CTP6 cards will be programmed to cycle through data downloaded
into its memory on thirty six links at 6.4 Gbps.  Therefore, these three
cards emulate the entire 18-crate RCT with oRSC output.  A fourth CTP6
card, and at later stages CTP7 or MP7 card, will be used to implement
the Stage-1 algorithms.  

The firmware will be designed such that the core operation of the cards
is separated from a sandbox environment in which the algorithms are 
implemented.  Various groups are expected to contribute to the sandbox
firmware which implements the algorithm.  This division also enables
staged development of the algorithms while the actual cards use in
the fourth stage are changed. 

The CTP7 firmware for the readout of RCT and RCT testing firmware will
also be implemented at Wisconsin to test the full functionality for operations
in P5.

The oRSC to CTP6 demonstrator is used to verify the oRSC operation at
Wisconsin.  One partially filled test RCT crate will be used to drive
the oRSC so that semi-realistic operation of the system is verified.
Additional inputs to the CTP6 can be emulated by prototype other CTP6
cards as described in the 3-to-1 demonstrator.  When the CTP7 card is
available the test are repeated to verify interoperability of the
cards.

The oRSC to MP7 demonstrator will be used to verify their
interoperability in Prevessin.  Two RCT crates with two oRSCs are
available.  Additional inputs to MP7 at 6.4 Gbps can be emulated by
other CTP6/CTP7 cards so that mixed mode operation required for
the HF region, in any case, are tested fully at Prevessin.

The MP7 in this case can also function as an algorithm testbed
at Prevessin.  We envision that the algorithm firmware developers
will be able to use this system to fully validate them before
deployment at P5.


The Prevessin environment will also have an ECAL TCC setup with
oSLBs, RCT with oRMs, HF $\mu$TCA prototypes.  We propose to 
hook them up to one MP7 board using oRSCs and CTPs.  This provides
a complete vertical slice test.  Such a test with 5-10\% of the
channels instrumented completely is necessary before the actual
system in P5 is dismantled for upgrade of oSLB/oRM system.


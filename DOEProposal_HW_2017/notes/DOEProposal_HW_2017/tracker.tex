\subsection{Silicon Strip Tracker\label{subsec:SiTracker}}
In standard p-p collision operation, the tracker detector sends the full detector information to 439 Front-End Devices (FEDs). They apply the common mode noise subtraction and the strip zero suppression. However, the common noise subtraction algorithm implemented in the Tracker FEDs firmware doesn't allow compensation of the baseline distortion observed in presence of Highly Ionizing Particles (HIP) with the consequent potential loss of clusters associated with the interested readout chip named APV. Due to the stringent requirements on tracking for high multiplicity HI events, the cluster finding inefficiency propagates directly to the track finding inefficiency depending on the centrality of the collision.  

The solution adopted up to now was to bypass the tracker zero suppression implemented in the tracker FEDs and send the full detector information to the DAQ/HLT. A specific HI common mode noise subtraction and zero suppression algorithms were implemented as HLT process. However, this solution implies a heavy load on the links between the FEDs and the DAQ and on the DAQ itself. The links are based on S-Link technology and with a maximum speed of 200 MB/s. Some FEDs (connected to sensors in which higher multiplicity is expected) have duplicated links allowing a maximum of $\sim 400$ MB/s per FED.  Considering that in average fragment size it is at ~32 kB/event. The readout limit is at around 12 kHz. It was also already demonstrated that the FED itself is designed to sustain up to 520 MB/s of throughput. The limitation is then only coming from the links to the DAQ. 

In agreement with the Tracker group, the only solution available to significantly increase the overall L1 rate is to modify the FEDs firmware including a more refined common mode subtraction algorithm accounting for baseline distortion. The output of the FED would be again zero suppressed allowing a reduction of the event size to DAQ of a factor of at leat 4. The system could then run at an L1 rate greater than 50 kHz. 

The tracker FEDs are based on Xilinx Virtex-2 FPGA that is quite limited in term of on-chip resources as well as the clock frequency. This limitation implies that only algorithm designed with specifically for HI could be implemented in the FEDs FPGA. Due to the limited on-chip resources, it would not be possible to have a unique FW version allowing p-p and Pb-Pb operation. Two FW versions are then expected and used accordingly depending on the type of collision system considered. 

In 2013 a specific HI zero suppression algorithm designed for high multiplicity environment and that would satisfy the FPGA constraints was designed offline and it is named a ``baseline follower''.  Preliminary studies performed by the tracker and HI groups showed promising performance results for the algorithm. The first part of the project requires the detailed study of the algorithm offline. It will imply roughly 3 months of work and detailed clustering and tracking studies. Hardware feasibility studies are going to be performed in parallel by tracker engineers. During this phase, the HI group should participate in providing the required information needed for the simulations. In the late spring/early summer 2017, the project will be handed over completely to the HI group where a developer will implement the actual algorithm in firmware. We estimate that the activity of coding will last 9 months. Afterward, three months commissioning time including tests on the bench and test at P5 should be considered. This schedule should leave some resources available for the actual 2018 HI run preparation. At present, the tracker group is focusing on the tracker operation during p-p collisions and the tracker upgrade project and a significant fraction of the work required for this project will need to fall under the HI group responsibility as mentioned above. 


For completeness, we should also report that other possible strategies were studied/considered to increase the L1 rate without modifying the tracker FED FW. However, they were rejected being either too complicated to be implemented or being underperforming. The two other solutions envisaged were either to increase the number of links between the tracker FEDs and the DAQ or to reduce the data payload. The first solution appears to be too complicated adding extra $\sim 400$ links. Apart from the technical constraints, there would also not be enough time to install and commission the links during the available shutdown time. The second solution is to reduce the data payload. A FED firmware version implemented already in 2015 allows the reduction from 10 to 8 the number of bits read-out by each strip ADC. However, this strategy would allow a moderate increase of $\le 20 \%$ of the overall L1 rate at the cost of a reduction of the detector resolution/sensitivity. 

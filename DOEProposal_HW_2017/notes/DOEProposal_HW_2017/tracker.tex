\subsection{Silicon Strip Tracker\label{subsec:SiTracker}}
In standard p-p collision operation, the tracker detector sends the full detector information to 430 Front-End Driver (FEDs). They apply common mode noise subtraction and strip zero-suppression. However, the common noise subtraction algorithm implemented in the Tracker FEDs firmware doesn't allow compensation of the baseline distortion observed in the presence of Highly Ionizing Particles (HIP), with the consequent potential loss of clusters associated on the affected readout chip. Due to the stringent requirements on tracking for high multiplicity HI events, the cluster-finding inefficiency propagates directly to the track-finding inefficiency, depending on the centrality of the collision.  

The solution adopted up to now was to bypass the tracker zero-suppression implemented in the tracker FEDs and send the full detector information to the DAQ/HLT. HI-specific common mode noise subtraction and zero-suppression algorithms were implemented as HLT processes. However, this solution results in a heavy load on the links between the FEDs and the DAQ and on the DAQ itself. The links are based on S-Link technology with a maximum speed of 200 MB/s. Some FEDs (connected to sensors in which higher multiplicity is expected) have duplicate links, allowing a maximum of $\sim 400$ MB/s per FED.  Considering that on average fragment size is ~32 kB/event, the readout limit is at around 12 kHz. It was also already demonstrated that the FED itself is designed to sustain up to 520 MB/s of throughput. The limitation is then only coming from the links to the DAQ. 

In agreement with the Tracker group, the only solution available to significantly increase the overall L1 rate is to modify the FEDs firmware, including a more refined common mode subtraction algorithm accounting for baseline distortion. The output of the FED would agin be zero-suppressed, allowing a reduction of the event size to DAQ of a factor of at least 4. The system could then run at an L1 rate greater than 50 kHz. 

The tracker FEDs are based on a Xilinx Virtex-2 FPGA that is quite limited in terms of on-chip resources as well as the clock frequency. This limitation implies that only an algorithm designed specifically for HI could be implemented in the FEDs' FPGA. Due to the limited on-chip resources, it would not be possible to have a single FW version allowing both p-p and Pb-Pb operation. Two FW versions would then be used, depending on the type of collision system considered. 

In 2013 a specific HI zero-suppression algorithm designed for high multiplicity environments and that would satisfy the FPGA constraints was designed offline; it is named a ``baseline follower''.  Preliminary studies performed by the Tracker and MIT HI groups showed promising performance results for the algorithm. The first part of the proposed project requires the detailed study of the algorithm offline. This will require roughly 3 months of work by a MIT student and detailed clustering and tracking studies founded by this proposal. We expect that hardware feasibility studies will be performed in parallel by tracker engineers using information provided but the MIT HI group. In late spring/early summer 2017, we anticipate the project will be handed over completely to the MIT HI group, where a developer (an engineer or a postdoctoral associate/scientist with hardware experience) will implement the actual algorithm in firmware. We estimate that the activity of coding will last 9 months. Afterward, three months of commissioning time, including tests on the bench and at P5, should be considered. This schedule should leave some resources available for the actual 2018 HI run preparation. The 12 months of work are expected to be founded by this proposal.

For completeness, we should also report that other possible strategies were studied/considered to increase the L1 rate without modifying the tracker FED FW. However, they were rejected as being either too complicated to be implemented or as underperforming. The two other solutions envisaged were either to increase the number of links between the tracker FEDs and the DAQ or to reduce the data payload. The first solution appears to be too complicated, adding an extra $\sim 400$ links. Apart from the technical constraints, there would also not be enough time to install and commission the links during the available shutdown time. The second solution is to reduce the data payload. A FED firmware version implemented already in 2015 allows the reduction from 10 to 8 in the number of bits readout by each strip ADC. However, this strategy would allow only a moderate increase of $\le 20 \%$ of the overall L1 rate at the cost in a reduction of the detector resolution/sensitivity. 

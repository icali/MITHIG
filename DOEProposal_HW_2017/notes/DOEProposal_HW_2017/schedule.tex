\section{Management Plan, Personnel, Schedule}
\label{sec:management}

The overall management of the upgrade of the L1 calorimeter trigger will be conducted within the existing management structure of CMS. MIT personnel supported by this request will be part of the CMS upgrade project and following the established management procedures within CMS. MIT group will be collaborating with the groups from Rice University, University of Wisconsin and Imperial College, London. The leader of the project related to the trigger functionality for the heavy ion physics will be Dr. Yen-Jie Lee of CERN \& MIT, who is presently the upgrade coordinator within the CMS heavy ion group. He will be part of the larger L1 Calorimeter Upgrade team led by Alex Tapper of Imperial College, who is in turn member of the overall CMS Upgrade project led by Jeff Spading and Didier Contardo.

The responsibilities related of the heavy ion group of the trigger upgrade will be the following:

\begin{itemize}
\item Establishment of the detailed triggering specifications based on the requirements of heavy ion physics;
\item Purchase and testing of the 3 prototype oRSC boards;
\item Testing of the boards and firmware at the trigger demonstrator at CERN/Prevessin CMS electronics testing laboratory;
\item Development of FPGA firmware on MP7 boards with specific heavy-ion triggering algorithms;
\item Purchase and testing of the final 22 oRSC boards;
\item Purchase of optical fibers and patch panel;
\item Installation and commissioning of the new trigger in CMS experiment.
\end{itemize}

The results of the physics studies as described in section~\ref{sec:physics} will be translated to firmware requirements by the MIT and Wisconsin groups.

The electronic boards will be designed by the electronics facility at the University of Wisconsin. They will supervise production of the prototypes and the final boards. The boards will be extensively tested at CERN/Prevessin CMS electronics testing laboratory by Ivan Cali and two graduate students under the supervision of Pamela Klabbers of Wisconsin. 

The firmware for the heavy ion calorimeter trigger will be developed by Ivan Cali of MIT in collaboration with Imperial College group. 

The installation and commissioning of the system will be done in the CERN/P5 CMS experimental hall by the MIT, Rice, Wisconsin and Imperial College groups.

The schedule of the project is driven by the need to install the system before the resumption of operations presently expected at the beginning of 2015. We identified the following milestones in the proposed schedule:

\begin{itemize}
\item Simple test code running on the MP7 firmware June 2013;
\item Prototype oRSC board arrives at the CERN testing facility: June 2013;
\item Testing of the oRSC board finished August 2013;
\item Production of the 22 boards starts, September 2013;
\item First version of a functional trigger firmware available for testing, December 2013;
\item Installation of the boards in CERN/P5 completed, May 2014;
\item Commissioning of the system in CERN/P5, December 2014;
\end{itemize}

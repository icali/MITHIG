\section{Management Plan, Personnel, Schedule}
\label{sec:management}

The overall management of the upgrade of the tracker and ECAL for the 2018 HI run will be conducted within the existing management
structure of CMS. MIT personnel supported by this request will be part of the two main groups and following the
established management procedures within CMS. The leader of the project related to the trigger functionality for
the heavy ion physics will be Dr. Yen-Jie Lee of CERN \& MIT, who is presently the upgrade coordinator within the CMS
heavy ion group. 
The responsibilities related of the heavy ion group in the various projects mentioned above will be the following:

\begin{itemize}
\item Help and support to the pixel group for the validation and commissioning of the pixel FEDs for HI collisions 
\item Detailed performance studies of the alternative algorithm for the tracker zero suppression. It includes the
commissioning of the offline software dedicated to the local reconstruction code

\item Re-design of the tracker FED firmware including the derivative follower algorithm
\item Participate in the phases of validation, test and commissioning of the HI tracker FED firmware in the CMS experiment
\item Performance sties on ECAL data reduction adjusting ECAL operational conditions. This activity implies also a
significant re-design of the offline ECAL local reconstruction
\item Adjust/re-design a part of the ECAL DCC firmware to increase overall rate. Also small hardare changes including power distributions/supply should be envisaged. 
\item Performance studies, test and commissioning of the specific HI L1 algorithms 
\item Development and commissioning of a dataflow able to reduce the HLT output event size

\end{itemize}

At present the performance physics studies required in the early phases of the project are carried by MIT graduate
students and a CERN porstdoc under the supervision of Ivan Cali. 

The pixel FED performance studies are going to be performed by the William Johns and Karl Ecklund. The contact person
for the HI group is still Ivan Cali. 
 
The firmware for the silicon tracker will be developed by Ivan Cali of MIT in collaboration with another person in the HI community and the Imperial
College group.

The firmware for the ECAL DCC change doesn't have a person associated yet. The LIP group will provide support.

The L1 trigger firmware implementation will be performed by the Imperial College group. At present the L1 contacts are: Maxime Guilbaud and Ivan Cali. 

The schedule of the whole project/s is driven by the need to terminate the project/s by the beginning of fall 2018. Few months are left for online operation and final commissioning phases before the 2018 HI run. 
We identified the following milestones in the proposed schedule:

\begin{itemize}
\item Silicon Pixel FED performance to be evaluated by June 2017;
\item Silicon Tracker zero suppression algorithm performance studies to be fineshed by May/June 2017;
\item Silicon Tracker FW implementation to be completed by April 2018;
\item Commissioning of the Silicon Tracker FED and corresponding software, September 2018;
\item ECAL preliminare performance studied to be concluded by June 2017;
\item ECAL calibration and corresponding offline software by August 2018;
\item ECAL firmware implementation by March 2018;
\item ECAL DCC commissioning finished by September 2018;
\item Commissioning of the system in CERN/P5, September 2018;
\item L1 trigger specific algorithms full definition by September 2017;
\item L1 trigger algorithm  implementation by March 2018;
\end{itemize}

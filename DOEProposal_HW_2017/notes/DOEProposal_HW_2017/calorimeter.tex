\subsection{ECAL\label{subsec:ECAL}} 
The ECAL off-detector electronics (DCC) was designed to cope with an L1 rate of 100 kHz and an average event size of 2 kB/event. During p-p high pile-up collisions the average event size is at around 1.5-1.8 kB. However, during Pb-Pb collisions the average event size is 15-20 kB (depending on the trigger mix) while the average event size for central Pb-Pb collisions is 34 kB. With such large event sizes, the maximum tolerated L1 rate is 12 kHz. In the specific ECAL case, hardware limits can be identified in two main areas: the link between ECAL DCC and DAQ, and the internal DCC speed. The links to DAQ are standard S-Link designed for 200 MB/s. For the 2017 run, an upgrade of the mezzanine card on the DCC side including a bigger buffer is already planned. This upgrade should allow an increase of 10\% in the total L1 rate. Also, the increase of clock frequency on the mezzanine card is being considered. The second limit encountered is the actual DCC internal firmware speed. An increase of 50 \% (and perhaps 100 \%) in speed can be achieved by modifying the internal DCC firmware structure, allowing the firmware to run at a higher clock frequency. 

The ECAL group engineers already expect to participate and supervise the project. However, the concrete implementation should be the responsibility of the CMS Heavy Ion group. A person should be dedicated to the project for roughly 6 months; funding is requested for this person.

In parallel to the DCC firmware development project, another project adjusting the operational DCC parameters is anticipated. An extra reduction in the event size can be obtained by adjusting the Selective Readout and Zero Suppression threshold. The ECAL group is already studying the performance for p-p collisions and the study should be repeated by the HI group for Pb-Pb collisions. The ECAL event size could also potentially be reduced up to 40 \% by reducing the number of samples collected for each event. At present, the ECAL reads out 10 BXs for each event and this number can be programmed down to a minimum of 6 BXs. However, reducing the number of samples could have strong implications for the energy resolution. A complete study should be performed to understand the significance of the effect on the x-fit in Pb-Pb collisions and the impact on the physics program. Any of the two offline solutions mentioned above would require specific performance studies and also the implementation of CMSSW code for the new configuration. The calorimeter calibrations should also be re-derived. The work needed for the project of adjusting the operational DCC parameters could be done by the CMS HI group as part of the CMS Experimental Physics Responsibility (EPR) that corresponds to the service work that should be provided by each CMS author.  





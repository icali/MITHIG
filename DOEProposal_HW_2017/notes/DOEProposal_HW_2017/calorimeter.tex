\subsection{ECAL\label{subsec:ECAL}} 
The ECAL off-detector electronics (DCC) was designed to cope with an L1 rate of 100 kHz and an average event size of 2 kB/event. During p-p high-pile-up collisions the average event size is at around 1.5-1.8 kB. However, during Pb-Pb collisions, the average event size is of 15-20 kB (depending on the trigger mix) while the average event size for central PbPb collisions is 34 kB. With so big event sizes, the L1 rate tolerated is up to 12 kHz. In the specific ECAL case, the hardware limits can be identified in two main areas: the link between ECAL DCC and DAQ, and the internal DCC speed. The links to DAQ are standard S-Link designed for 200 MB/s. For the 2017 run, an upgrade of the mezzanine card on the DCC
side including a bigger buffer is already planned. This upgrade should allow an increase of 10\% in the total L1 rate. Also, the increase of clock frequency on the mezzanine card is being considered. The second limit encountered is the actual DCC internal firmware speed. An increase of 50 \% (maybe also 100 \%) in speed can be achieved modifying the internal DCC firmware structure allowing the firmware to run at a higher clock frequency. 

The ECAL group engineers gave already their availability to participate and supervise the project. However, the concrete implementation should be the responsibility of the CMS Heavy Ion group. A person should be dedicated to the project for roughly 6 months. 

In parallel to the DCC firmware development project, another project adjusting the operational DCC parameters is started. An extra reduction on the event size can be obtained adjusting the Selective Readout and Zero Suppression threshold. The ECAL group is already studying the performance for p-p collision and the study should be repeated by the HI group for Pb-Pb collisions. On the same line, a part of the project consists also in reducing the ECAL event size (potentially up to 40 \% reduction) reducing the number of samples collected for each event. At present, the ECAL reads out 10 BX for each event
and this number can be programmed down to a minimum of 6 BX. However, reducing the number of samples could have strong implication in the energy resolution. A complete study should be performed to understand how significant could be the effect of x-fit in Pb-Pb collisions and the impact on the physics program. Considering the amount of work required for the offline studies mentioned above, we should allocate a person for one year to the project. Any of the two offline solutions mentioned above indeed require specific performance studies and also the implementation of CMSSW code for the eventually new configuration. The calorimeter calibrations should also be re-derived. 





\newpage

\section{ECAL\label{sec:ECAL}} 
The ECAL off detector electronics (DCC) was designed to cope with a L1 rate of 100 kHz and an average event size of 2
kB/event.  During p-p high-pu collisions the average event size is at around 1.5 – 1.8 kB/event. However, during Pb-Pb
collisions, the average event size is of 15-20 kB/event (depending by the trigger mix) and it is up to 34 kB/event for
central PbPb collisions. With so big event sizes, the L1 rate tolerated is up to 12 kHz. In the specific ECAL case, the
hardware limits can be identified in two main areas: the link between ECAL DCC and DAQ, and the internal DCC speed.  The
links to DAQ are standard S-Link designed for 200 MB/s. For the 2017 run an upgrade of the mezzanine card on the DCC
side including a bigger buffer is already planned. This upgrade should allow an increase of 10% in the total L1 rate.
Also the increase of clock frequency on the mezzanine card could be considered but the benefits are still to be
evaluated. The second limit encountered is the actual DCC internal speed firmware speed. An increase of 50 % (maybe also
100 %) in speed can be achieved modifying the internal DCC firmware structure allowing the firmware to run at a higher
clock frequency. The ECAL group engineers gave already their availability to participate and supervise the project.
However, the concrete implementation should be responsibility of the HI group. A person should be dedicated to the
project for roughly 6 months. Before envisaging the firmware change hypothesis, there are other two configurations that
should be evaluated. The ECAL group is already studying the calorimeter performance adjusting the Selective Readout and
Zero Suppression threshold. It is not clear yet how much the new settings could influence the event size for Pb-Pb
collisions but definitively a benefit could be directly obtained. On the same line, the ECAL event size could be reduced
up to 40 % reducing the number of samples collected for each event. At present the ECAL reads out 10 BX for each event
and this number can be programmed down to a minimum of 6 BX. However, reducing the number of samples could have strong
implication in the energy resolution. A complete study should be performed to understand how significant could be the
effect of x-fit in Pb-Pb collisions and the consequently effects on the physics program. Considering the amount of work
required for the offline studies mentioned above, we should allocate a person for one year to the project. Any of the
two offline solutions mentioned above indeed require specific performance studies and also the implementation of CMSSW
code for the eventually new configuration. Also the calorimeter calibrations should be re-derived. 





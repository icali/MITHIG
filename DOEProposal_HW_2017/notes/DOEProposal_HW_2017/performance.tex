\section{Heavy Ion Trigger Performance} \label{sec:hiTrigPerf}

In this Section we discuss detailed jet trigger performance studies comparing the 
current L1 trigger system as described in section~\ref{calo:present},
the Stage-1 upgrade described in Section~\ref{calo:stage1} and 
the Stage-2 upgrade  planned for the
high luminosity LHC (HL-LHC) with much higher granularity of inputs from
calorimeter cells. 
For all systems the trigger performance was evaluated based on stored 
information (``trigger primitives'') from real data 
collected in the 2011 heavy ion data taking period.
Using real data is preferable for  L1 trigger efficiency estimation since 
the effects of underlying event are not simulated perfectly by MC
simulations. 

\subsection{Jet Trigger Requirements}
One key measure of the performance of a trigger is the rejection rate and the
response of the rejection rate to increased threshold of the trigger. 
Another measure of the trigger performance  is how efficient the
trigger is at selecting jets above its threshold and rejecting those below
the threshold; this is a jet turn-on curve. In a plot of accepted jets as a
function of offline jet \pt\ (iterative cone calorimeter jets for this
study), an effective trigger should produce a sharp turn-on from 0\% accepted to
100\% accepted. The offline jet \pt\ value at which the trigger is 100\%
efficient will not perfectly match the threshold value because of limitations in the
jet definition at L1; the figure of merit is the steepness of the curve.


Figure \ref{fig:efficiency_comparison} shows a comparison of the trigger
systems' rejection rate as a function of the threshold set at L1. 
With the luminosity increase after long shutdown LS1, an accept rate of 5\% (rejection
factor of 20) is required in order to remain sample dead-time free. This would allow 
sampling of the full delivered luminosity for the desired jet trigger paths 
at the HLT. 
%\begin{figure}[!ht]
%\begin{center}
%\includegraphics[width=.60\textwidth]{fig/heavyIon//efficiency_comparison_l1primitives.pdf}
%\caption{The jet finder is applied to minimum bias data from 2011 without
%event selection using the L1 primitives present in the RAW data. The
%fraction of events which have a jet above a given $E_t$ threshold is plotted
%as a function of that threshold. Note that the threshold is applied to the
%L1 jet energy and does not correspond to the HLT or offline jet energy
%scales.}
%\label{fig:efficiency_comparison}
%\end{center}
%\end{figure}


In the case of heavy ion collisions, the assessment of the trigger performance
as a function of collision centrality is important. 
A trigger may be effective at low centrality (low multiplicity) but
inefficient in very central events, i.e., at high multiplicity. 

The jet turn-on curves displayed in the next sections are created using L1 primitives and
minimum bias data with event selection. 
Only events without beam halo and with $>$ 3 GeV in each forward calorimeter
are used. The sample is broken into two parts, 
one with centrality less than 30\% and the other greater than 50\%. For each
event, the L1 jet finder is run and the given threshold is applied. 
The fraction of events which pass the threshold as a function of the highest
offline jet \pt\ 
%(icPu5CaloJets)  Not defined and jargon anyway -Matt
 is plotted. If points are missing from an image, this means that 0 events. The figures contain
 turn-on curves for two different jet thresholds.
%were accepted in that bin.

\begin{figure}[htbp]
\begin{center}
\includegraphics[width=.60\textwidth]{fig/heavyIon//jetto_current.pdf}
\caption{Jet trigger turn-on curve for the current system at L1 thresholds of 60 and 100
GeV. Results are shown for peripheral events (open symbols) and central events (filled symbols).
A large shift in the turn-on curve as a function of centrality is seen.}
\label{fig:jetto_current}
\end{center}
\end{figure}

\begin{figure}[htbp]
\begin{center}
\includegraphics[width=.60\textwidth]{fig/heavyIon//jetto_2015.pdf}
\caption{Jet trigger turn-on curve for the Stage-1 system at L1 thresholds of 60 and 100
GeV. Results are shown for peripheral events (open symbols) and central events (filled symbols).
Full efficiency is reached at the same \pt\ for central and peripheral events.}
\label{fig:jetto_2015}
\end{center}
\end{figure}

\subsection{Existing system}

As can be seen from Figure \ref{fig:efficiency_comparison}, the existing
trigger system requires an unacceptably high threshold of $p_{\rm T}\, \gtrsim 350$~GeV to 
reach the 5\% accept rate required for the luminosity increase. Such a
threshold would severely limit the physics capabilities of the detector. 

Even for lower thresholds (with high dead-time) in the high luminosity case),
Figure~\ref{fig:jetto_current} shows that the current system performs poorly 
for high centrality events. Comparing central to peripheral events, a large
shift in the turn-on curve is seen, when comparing the same nominal threshold
for central and peripheral events.

\subsection{Stage-1 system}

Figure \ref{fig:efficiency_comparison} shows that the required threshold for
the Stage-1 system is relatively low at about $p_{\rm T}\, \sim 75$~GeV, 
comparable to the full Stage-2 system. Also at even lower \pt\ the Stage-1 system
has a much smaller accept fraction than the current system, allowing smaller 
prescale factors while preserving the bandwidth constraints.
The jet turn-on curves, Figure~\ref{fig:jetto_2015} 
shows that the trigger is efficient and has similar performance for the
high and low centrality samples, 
especially at the lower threshold. The Stage-1 system would adequately meet the
requirements of the Heavy Ion program at the increased luminosity.

\subsection{Stage-2 system}

The most effective trigger, the full stage-2 system is even more responsive
to threshold changes 
than the stage-1 system and shows nearly identical performance between the two
centrality samples. 
The Stage-2 allows for more flexible L1 trigger jet algorithms as well,
including a single-tower trigger which could be used to look for high-\pt\
tracks to be used as seeds for other physics objects.


\section{Management Plan, Personnel, Schedule}
\label{sec:management}

The overall organization of the upgrade effort of the tracker% and ECAL 
for the 2018 HI run will be conducted within the existing management structure of CMS. MIT personnel are part of the two corresponding detector groups and follow the established management procedures within CMS. The leader of the project for Heavy Ion physics will be Prof. Yen-Jie Lee of MIT. The project will include the following deliverables:

\begin{itemize}
\item Detailed performance studies of the alternative algorithm for the tracker zero suppression. It includes the commissioning of the offline software dedicated to the local reconstruction code;
\item Re-design of the tracker FED firmware including the derivative follower algorithm;
\item Participation in the phases of validation, test and commissioning of the HI tracker FED firmware in the CMS experiment;
\end{itemize}

The performance physics studies required in the early phases of the project are already being carried out by MIT graduate students under the supervision of Dr. Ivan Cali. The firmware for the silicon tracker will be developed by a engineer who is part of the HI group, supervised by Dr. Ivan Cali of MIT. The project will be closely coordinated with the experts at the Imperial College, London group.

The schedule of the whole project is driven by the need to complete the project by the beginning of Fall 2018. Then a few months are left for online operation and final commissioning phases before the 2018 HI run. We have identified the following milestones in the proposed schedule:

\begin{itemize}
\item Silicon Tracker zero suppression algorithm performance studies to be finished by July 2017;
\item Silicon Tracker FW implementation to be completed by June 2018;
\item Commissioning of the Silicon Tracker FED and corresponding software by September 2018;
\item Commissioning of the system in CERN/P5, October 2018;
\end{itemize}

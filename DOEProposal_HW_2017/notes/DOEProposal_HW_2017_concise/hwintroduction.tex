The CMS experiment is mainly designed to record high luminosity, high pile-up (PU) proton-proton collisions events.  With several detector adjustments/configuration changes over the last several years, the CMS experiment has been able to record successfully also the high multiplicity heavy ion collisions delivered by the LHC. CMS ran smoothly during the 2015 Pb-Pb data-taking with an average L1 rate of $\sim 10$ kHz. During the run, a few hours of the Pb-Pb data-taking period were devoted to exploring the maximum L1 rate limit. With the 2015 CMS configuration for Pb-Pb, the absolute L1 rate limit was observed at 12 kHz. 

During the 2015 run, the Pb-Pb collision rate was $\sim 20$ kHz. For the 2018 Pb-Pb run we expect an increase in the collision rate to $\sim 30$ kHz. Without applying any changes to the L1 trigger mix used in 2015, this rate increase would mean operating CMS above its limit, with a significant impact on the heavy ion physics program. In addition, it would be beneficial to also further increase the number of minimum bias events collected for detector calibration, low \pt jet and heavy flavor physics. 

The CMS subsystems, as well as their interaction with the DAQ, have been studied and the most significant bottleneck area, the Tracker FED FW upgrade, has been identified. This update project is a perfect match with MIT expertise, for which funds are requested in this supplement proposal. This proposal focuses on aspects that will require hardware changes. 


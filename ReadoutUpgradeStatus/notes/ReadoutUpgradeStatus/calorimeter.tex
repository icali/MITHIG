\subsection{ECAL\label{subsec:ECAL}} 
The ECAL off-detector electronics (DCC) were designed to cope with an L1 rate of 100 kHz and an average event size of 2 kB/event. 
During p-p high pile-up collisions the average event size is at around 1.5-1.8 kB. However, during Pb-Pb collisions the average event size is 15-20 kB (depending on the trigger mix) while the
average event size for central Pb-Pb collisions is 34 kB. With such large event sizes, the maximum tolerated L1 rate is 12 kHz. In order to achieve the desired rate of 30 kHz it is needed to either modify the hardware to increase the overall bandwith or to reduce the data volume to DAQ. 


In the specific ECAL case, hardware limits can be identified in two main areas: the link between ECAL DCC and
DAQ, and the internal DCC speed. The links to DAQ are standard S-Link designed for 200 MB/s. For the 2017 run, an
upgrade of the mezzanine card on the DCC side including a bigger buffer is already planned. This upgrade should allow an
increase of 10\% in the total L1 rate. Also, the increase of clock frequency on the mezzanine card is being considered.
The second limit encountered is the actual DCC internal firmware speed. An increase of 50 \% (and perhaps 100 \%) in
speed can be achieved by modifying the internal DCC firmware structure, allowing the firmware to run at a higher clock
frequency.

An event size reduction can be obtained by adjusting the Selective Readout and Zero
Suppression threshold. The ECAL group has been studied extensively this possibility and in p-p collisions the event size reduction is significant without any physics loss. The studies were repeated in HI environment using both MC events and re-emulating the ECAL in 2015 Pb-Pb data. 

The comprensive presentation showing the results obtained from the studies can be found in \cite{ECALDPG} and \cite{ECALHIN}. In summary, two differe



The ECAL event size could also potentially be reduced up to 40 \% by
reducing the number of samples collected for each event. At present, the ECAL reads out 10 BXs for each event and this
number can be programmed down to a minimum of 6 BXs. However, reducing the number of samples could have strong
implications for the energy resolution. 







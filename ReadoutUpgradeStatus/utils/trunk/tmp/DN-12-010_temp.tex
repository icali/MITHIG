%%%%%%%%%%%%%%%%%%%%%%%%%%%%%%%%%%%%%%%%%%%%%%%%%%%%%%%%%%%%%%%%%%%%
%
%   Style for CMS Computing / Physics Technical Design Reports
%
%   Lucas Taylor  4 Feb 2005,   Revised  12 Oct 2005
%
%%%%%%%%%%%%%%%%%%%%%%%%%%%%%%%%%%%%%%%%%%%%%%%%%%%%%%%%%%%%%%%%%%%%

%  the following line is edited by the tdr script to change or to pass
%  additional options:
\documentclass[11pt,twoside,a4paper,note]{cms-tdr}
\def\svnVersion{201321}

%%%%%%%%%%%%%%%%%%%%%%%%%%%%%%%%%%%%%%%%%%%%%%%%%%%%%%%%%%%%%%%%%%%%

\begin{document}
%%%%%%%%%%%%%%%%%%%%%%%%%%%%%%%%%%%%%%%%%%%%%%%%%%%%%%%%%%%%%%%%%%%%
%
%  Common definitions
%
%  N.B. use of \providecommand rather than \newcommand means
%       that a definition is ignored if already specified
%
%                                              L. Taylor 18 Feb 2005
%%%%%%%%%%%%%%%%%%%%%%%%%%%%%%%%%%%%%%%%%%%%%%%%%%%%%%%%%%%%%%%%%%%%


%%%%%%%%%%%%%%%%%%%%%%%%%%%%%%%%%%%%%%%%%%%%%%%%%%%%%%%%%%%%%%%%%%%%
%
% Hyphenations (only need to add here if you get a nasty word break)
%
\hyphenation{had-ron-i-za-tion}
\hyphenation{cal-or-i-me-ter}
\hyphenation{de-vices}
%
% Hyphenations-end
% % Customizable fields and text areas start with % >> below.
% Lines starting with the comment character (%) are normally removed before release outside the collaboration, but not those comments ending lines

% svn info. These are modified by svn at checkout time.
% The last version of these macros found before the maketitle will be the one on the front page,
% so only the main file is tracked.
% Do not edit by hand!
\RCS$Revision: 167902 $
\RCS$HeadURL: svn+ssh://svn.cern.ch/reps/tdr2/notes/DN-12-010/trunk/DN-12-010.tex $
\RCS$Id: DN-12-010.tex 167902 2013-01-28 08:49:11Z yjlee $
%%%%%%%%%%%%% local definitions %%%%%%%%%%%%%%%%%%%%%
% This allows for switching between one column and two column (cms@external) layouts
% The widths should  be modified for your particular figures. You'll need additional copies if you have more than one standard figure size.
\newlength\cmsFigWidth
\ifthenelse{\boolean{cms@external}}{\setlength\cmsFigWidth{0.85\columnwidth}}{\setlength\cmsFigWidth{0.4\textwidth}}
\ifthenelse{\boolean{cms@external}}{\providecommand{\cmsLeft}{top}}{\providecommand{\cmsLeft}{left}}
\ifthenelse{\boolean{cms@external}}{\providecommand{\cmsRight}{bottom}}{\providecommand{\cmsRight}{right}}
%%%%%%%%%%%%%%%  Title page %%%%%%%%%%%%%%%%%%%%%%%%
\cmsNoteHeader{DN-12-010} 
\title{Performance Study of L1 Calorimeter Trigger for Heavy Ion Collisions with High Luminosity}%

% >> Authors
%Author is always "The CMS Collaboration" for PAS and papers, so author, etc, below will be ignored in those cases
%For multiple affiliations, create an address entry for the combination
%To mark authors as primary, use the \author* form
\author[cern]{The CMS Collaboration}

% >> Date
% The date is in yyyy/mm/dd format. Today has been
% redefined to match, but if the date needs to be fixed, please write it in this fashion.
% For papers and PAS, \today is taken as the date the head file (this one) was last modified according to svn: see the RCS Id string above.
% For the final version it is best to "touch" the head file to make sure it has the latest date.
\date{\today}

% >> Abstract
% Abstract processing:
% 1. **DO NOT use \include or \input** to include the abstract: our abstract extractor will not search through other files than this one.
% 2. **DO NOT use %**                  to comment out sections of the abstract: the extractor will still grab those lines (and they won't be comments any longer!).
% 3. **DO NOT use tex macros**         in the abstract: External TeX parsers used on the abstract don't understand them.
\abstract{
}

% >> PDF Metadata
% Do not comment out the following hypersetup lines (metadata). They will disappear in NODRAFT mode and are needed by CDS.
% Also: make sure that the values of the metadata items are sensible and are in plain text (no TeX! -- for \sqrt{s} use sqrt(s) -- this will show with extra quote marks in the draft version but is okay).

\hypersetup{%
pdfauthor={Alex Barbieri, Yen-Jie Lee, Wei Li, Matthew Nguyen, Christof Roland},%
pdftitle={Performance Study of L1 Calorimeter Trigger for Heavy Ion Collisions with High Luminosity},%
pdfsubject={CMS},%
pdfkeywords={CMS, physics, software, computing}}

\maketitle %maketitle comes after all the front information has been supplied
% >> Text
%%%%%%%%%%%%%%%%%%%%%%%%%%%%%%%%  Begin text %%%%%%%%%%%%%%%%%%%%%%%%%%%%%
%% **DO NOT REMOVE THE BIBLIOGRAPHY** which is located before the appendix.
%% You can take the text between here and the bibiliography as an example which you should replace with the actual text of your document.
%% If you include other TeX files, be sure to use "\input{filename}" rather than "\input filename".
%% The latter works for you, but our parser looks for the braces and will break when uploading the document.
%%%%%%%%%%%%%%%
\section{Introduction}
\label{sec:intro}

Relativistic heavy ion collisions allow experimental studies of quark-gluon plasma (QGP), the equilibrated high-temperature state of deconfined quarks and gluons. At the Large Hadron Collider (LHC), the abundant production of hard probes such as vector bosons, heavy quarks, jets, and quarkonia produced in the highest energy heavy ion collisions has opened a new era in the characterization of QGP, providing new information on initial state properties, parton energy loss and color screening. The capabilities of the Compact Muon Solenoid (CMS) detector in the detection of high momentum photons, electrons, muons, charged particles, and jets have proven to be a unique match for the opportunities at the LHC. A key aspect of this success has been the convergence of detector needs for precision measurements in pp discovery physics in a high luminosity, high pileup environment and for QGP studies in the high multiplicity PbPb collisions. This convergence has allowed CMS to adapt a large number of new analysis techniques to studies of heavy ion collisions, such as particle-flow jet reconstruction, studies of jet substructures, lifetime fits for secondary $J/\psi$ studies,  b- and c-tagging of jets, missing \pt measurements of W production and energy flow in dijet events. 

The full power of these and other techniques will be exploited in future heavy ion studies beginning in 2018 and Run III, when an increase in the collision energy to $\sqrt{s_{_{NN}}} = 5.02$~TeV and an eventual increase in PbPb collision rate to as high as $\sim 30-50$~kHz will enhance the production rate of hard probes by more than an order of magnitude. In this high energy and high luminosity era, CMS will undertake precision studies of complex observables such as $\gamma$-, Z$^0$- and W$^\pm$-jet correlations, heavy flavor jet quenching, multi-jet correlations and the azimuthal anisotropy of high \pt\ jets and quarkonia. Recently, extensive studies of fully reconstructed heavy flavor mesons such as $D^0$ and $B^+$ in heavy ion collisions also show that CMS is ideally suited for the studies of heavy flavor production and heavy quark energy loss in the quark-gluon plasma.

The CMS trigger and data acquisition system is the key to achieving the ultimate high \pt\ physics reach in heavy ion collisions. Uniquely, the CMS trigger system only has two main components, the hardware-based Level-1 (L1) trigger and the High-Level Trigger (HLT), which is implemented as "offline" algorithms running on a large computer farm with access to the full event information.  
The conceptual aspects of applying this system to heavy ion collisions, which are characterized by much lower rate, but much higher multiplicity than those encountered in pp collisions, were first developed by the US CMS HI group (see, e.g.\ \cite{Roland:2007is}). The HLT trigger work formed the basis for our previous proposals to the DOE Office of Nuclear Physics. As we demonstrated in that proposal, triggering on high \pt\ probes for PbPb collision rates up to several kHz (i.e.\ the design value for PbPb) is possible using event rejection solely or mostly at the HLT level. In the 2011 PbPb run, this allowed a reduction of the event rate to storage by more than an order of magnitude compared to the collision rate, without prescaling the most interesting high \pt\ observables. In the 2015 PbPb run, the heavy flavor meson triggers developed by the US CMS HI group from a previous DOE proposal was deployed for the first time. For instance, those triggers increased the high \pt\ $D^0$ meson statistics in pp (PbPb) collisions by a factor of 600 (20) which opened a new era for the precision heavy flavor physics in heavy ion collisions.

It has now become clear that the LHC will be able to significantly exceed the PbPb design luminosity in future runs, possibly reaching up to 50~kHz already in Run III. This will place an even greater emphasis on the CMS trigger and DAQ system. The CMS during PbPb collisions is being operated close to its hardware limit in term of readout rate (see sec.~\ref{sec:HWUpgrade} for details). As a consequence, the CMS will not be able to benefit of the increased luminosity without any dedicated upgrade. 
In 2017, a new 4-layer pixel detector will be installed in CMS which will greatly improve the impact parameter resolution of the charged tracks. This is an unique opportunity to combine the capability of CMS for the studies of high \pt\ probes with heavy flavor program down to \pt$\sim 0$, complimentary to the physics goal of the O(100M) ALICE upgrade. Moreover, a stage-2 upgrade of the Level-1 trigger system will be commissioned for PbPb data-taking. This requires improvement in the front-end detector readout bandwidth in PbPb collisions, as well as significant development on the trigger strategy for the data-taking in 2018 and beyond such that CMS could provide unbiased single track, heavy flavor meson, jet triggers, that are critical for the CMS physics program, and at the same time record a large statistics Minimum-Bias triggered sample which could be used to for the studies of low \pt\ heavy flavor mesons. This necessitates the development of underlying event background subtraction for the stage-2 L1 trigger system and improvements in the detector readout to increase the maximum L1 trigger accept rate during the PbPb data-taking period.

In this proposal, we will present the CMS L1 rate upgrade design and demonstrate that the planned upgrade delivers the needed capability to record heavy flavor meson and jets over a very wide kinematics range at the full delivered rate in high luminosity PbPb runs in 2018 and beyond. This new system will need to be commissioned by 2017-2018, before the next heavy-ion running period at the end of 2018, to make successful PbPb data-taking possible. Section~\ref{sec:physics} describes the expected performance of the upgraded trigger for several examples of critical measurements, compared to the limitations of the current system. The technical details of the proposed upgrade project are described in Sections~\ref{subsec:ECAL} and~\ref{subsec:SiTracker}. The support projects needed to guarantee an efficient L1 rate increase are described in Sections~\ref{subsec:SiPixel},~\ref{subsec:L1Trigger} and~\ref{subsec:dataflow}. Finally, the proposed schedule, management plan, and the budget are described in Sections~\ref{sec:management} and~\ref{sec:funding}.


\input{physicsMotivation}
%\input{currentSystem}
\input{upgradeSchemes}
\section{Heavy Ion Trigger Performance} \label{sec:hiTrigPerf}

In this Section we discuss detailed jet trigger performance studies comparing the 
current L1 trigger system as described in Section~\ref{calo:present},
the Stage-1 upgrade described in Section~\ref{calo:stage1} and 
the Stage-2 upgrade  planned for the
high luminosity LHC (HL-LHC) with much higher granularity of inputs from
calorimeter cells. 
For all systems the trigger performance was evaluated based on stored 
information (``trigger primitives'') from real data 
collected in the 2011 heavy ion data taking period.
Using real data is preferable for  L1 trigger efficiency estimation since 
the effects of underlying event are not simulated perfectly by MC
simulations. 

\subsection{Jet Trigger Requirements}
One key measure of the performance of a trigger is the rejection rate and the
response of the rejection rate to increased threshold of the trigger. 
Another measure of the trigger performance  is how efficient the
trigger is at selecting jets above its threshold and rejecting those below
the threshold; this is a jet turn-on curve. In a plot of accepted jets as a
function of offline jet \pt\ (iterative cone calorimeter jets for this
study), an effective trigger should produce a sharp turn-on from 0\% accepted to
100\% accepted. The offline jet \pt\ value at which the trigger is 100\%
efficient will not perfectly match the threshold value because of limitations in the
jet definition at L1; the figure of merit is the steepness of the curve.


Figure \ref{fig:efficiency_comparison} shows a comparison of the trigger
systems' rejection rate as a function of the threshold set at L1. 
With the luminosity increase after long shutdown LS1, an accept rate of 5\% (rejection
factor of 20) is required in order to remain sample dead-time free. This would allow 
sampling of the full delivered luminosity for the desired jet trigger paths 
at the HLT. 
%\begin{figure}[!ht]
%\begin{center}
%\includegraphics[width=.60\textwidth]{fig/heavyIon//efficiency_comparison_l1primitives.pdf}
%\caption{The jet finder is applied to minimum bias data from 2011 without
%event selection using the L1 primitives present in the RAW data. The
%fraction of events which have a jet above a given $E_t$ threshold is plotted
%as a function of that threshold. Note that the threshold is applied to the
%L1 jet energy and does not correspond to the HLT or offline jet energy
%scales.}
%\label{fig:efficiency_comparison}
%\end{center}
%\end{figure}


In the case of heavy ion collisions, the assessment of the trigger performance
as a function of collision centrality is important. 
A trigger may be effective at low centrality (low multiplicity) but
inefficient in very central events, i.e., at high multiplicity. 

The jet turn-on curves displayed in the next sections are created using L1 primitives and
minimum bias data with event selection. 
Only events without beam halo and with $>$ 3 GeV in each forward calorimeter
are used. The sample is broken into two parts, 
one with centrality less than 30\% and the other greater than 50\%. For each
event, the L1 jet finder is run and the given threshold is applied. 
The fraction of events which pass the threshold as a function of the highest
offline jet \pt\ 
%(icPu5CaloJets)  Not defined and jargon anyway -Matt
 is plotted. If points are missing from an image, this means that 0 events. The figures contain
 turn-on curves for two different jet thresholds.
%were accepted in that bin.

\begin{figure}[htbp]
\begin{center}
%\includegraphics[width=.60\textwidth]{fig/heavyIon//jetto_current.pdf}
\caption{Jet trigger turn-on curve for the current system at L1 thresholds of 60 and 100
GeV. Results are shown for peripheral events (open symbols) and central events (filled symbols).
A large shift in the turn-on curve as a function of centrality is seen.}
\label{fig:jetto_current}
\end{center}
\end{figure}

\begin{figure}[htbp]
\begin{center}
%\includegraphics[width=.60\textwidth]{fig/heavyIon//jetto_2015.pdf}
\caption{Jet trigger turn-on curve for the Stage-1 system at L1 thresholds of 60 and 100
GeV. Results are shown for peripheral events (open symbols) and central events (filled symbols).
Full efficiency is reached at the same \pt\ for central and peripheral events.}
\label{fig:jetto_2015}
\end{center}
\end{figure}

\subsection{Existing system}

As can be seen from Figure \ref{fig:efficiency_comparison}, the existing
trigger system requires an unacceptably high threshold of $p_{\rm T}\, \gtrsim 350$~GeV to 
reach the 5\% accept rate required for the luminosity increase. Such a
threshold would severely limit the physics capabilities of the detector. 

Even for lower thresholds (with high dead-time) in the high luminosity case),
Figure~\ref{fig:jetto_current} shows that the current system performs poorly 
for high centrality events. Comparing central to peripheral events, a large
shift in the turn-on curve is seen, when comparing the same nominal threshold
for central and peripheral events.

\subsection{Stage-1 system}

Figure \ref{fig:efficiency_comparison} shows that the required threshold for
the Stage-1 system is relatively low at about $p_{\rm T}\, \sim 75$~GeV, 
comparable to the full Stage-2 system. Also at even lower \pt\ the Stage-1 system
has a much smaller accept fraction than the current system, allowing smaller 
prescale factors while preserving the bandwidth constraints.
The jet turn-on curves, Figure~\ref{fig:jetto_2015} 
shows that the trigger is efficient and has similar performance for the
high and low centrality samples, 
especially at the lower threshold. The Stage-1 system would adequately meet the
requirements of the Heavy Ion program at the increased luminosity.

\subsection{Stage-2 system}

The most effective trigger, the full stage-2 system is even more responsive
to threshold changes 
than the stage-1 system and shows nearly identical performance between the two
centrality samples. 
The Stage-2 allows for more flexible L1 trigger jet algorithms as well,
including a single-tower trigger which could be used to look for high-\pt\
tracks to be used as seeds for other physics objects.


\input{tracktrigger}
\input{summary}

\clearpage

\bibliography{auto_generated}   % will be created by the tdr script.  

%\clearpage
%\appendix
%\input{appendices.tex}
\end{document}


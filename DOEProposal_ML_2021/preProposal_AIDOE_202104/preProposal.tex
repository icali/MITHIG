In this Section we discuss the physics performance of the L1 trigger rate upgraded system for the 2018 high luminosity Pb-Pb run using three examples of measurements with rare probes. The performance of the upgraded system, which uses background subtraction for L1 event rejection and provides access to the full expected luminosity of 3~nb$^{-1}$, is compared to that of the current system. These examples illustrate the large improvements in statistical precision afforded by the improved ability to select jet events and high \pt\ particles during high luminosity Pb-Pb collisions with L1 trigger rate upgrade. 

Below we discuss a selection of measurements that exploit the future high-luminosity data sets to provide important insights into the parton energy loss mechanism and heavy flavor particle production:

\begin{enumerate}
\item The flavor dependence of jet quenching is an important test ground for various parton energy loss models. 
Compared to light quarks, heavy quarks are expected to suffer from smaller radiative energy loss when passing through the medium because gluon radiation is suppressed at angles smaller than the ratio of the quark mass $M$ to its energy $E$~\cite{Dokshitzer:2001zm}. This effect (and its disappearance at high \pt) can be studied using charged particles, heavy flavor mesons and b-jets.
%\item Compared to quarks, gluons are expected to suffer a larger 
%energy loss because of the larger color factors. This effect can be studied using three-jet events.
\item Measurements of the charged particle and heavy flavor meson azimuthal anisotropy ($v_2$) at very high \pt\ give access to the in-medium path-length 
dependence of energy loss. The azimuthal anisotropy as a function of particle \pt\ also yields important information about the parton energy dependence of the energy loss.  At low \pt, hadronization of heavy quarks such as parton-to-jet fragmentation and the recombination of heavy quarks and light flavors has a significant impact on the magnitude of the heavy flavor meson $v_2$. 
\item Studies of jet-heavy flavor meson correlation could give important insights into the parton energy loss mechanism and for the studies of medium induced radiation and medium response to energetic heavy quarks.
\end{enumerate}

These and other measurements rely on a sufficiently selective jet and heavy flavor meson trigger to provide access to the full delivered luminosity. At the same time, a large minimum-bias sample is needed in order to access very low \pt\ heavy flavor mesons and to perform data-driven tracking and jet energy corrections.  To illustrate the impact of the L1 rate upgrade, data collected during the 2015 Pb-Pb run were used to estimate physics reach for a future high-luminosity run (assuming $L_{int} =$3~nb$^{-1}$) for these three physics cases. 



\clearpage


